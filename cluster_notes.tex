\documentclass{amsart}
\usepackage{amsmath,amssymb,latexsym,enumerate}


\newtheorem{theorem}{Theorem}[section]
\newtheorem{conjecture}[theorem]{Conjecture}
\newtheorem{corollary}[theorem]{Corollary}
\newtheorem{definition}[theorem]{Definition}
\newtheorem{lemma}[theorem]{Lemma}
\newtheorem{proposition}[theorem]{Proposition}
\newtheorem{example}[theorem]{Example}
\newtheorem{subexercise}{Exercise}[theorem]
\newtheorem{exercise}[theorem]{Exercise}
\theoremstyle{remark} 
\newtheorem{remark}[theorem]{Remark}
\numberwithin{equation}{section}


\newcommand{\cA}{{\mathcal{A}}}
\newcommand{\cF}{{\mathcal{F}}}
\newcommand{\cO}{{\mathcal{O}}}
\newcommand{\bff}{{\bf f}}
\newcommand{\x}{{\bf x}}
\newcommand{\bx}{{\bf x}}
\newcommand{\y}{{\bf y}}
\newcommand{\by}{{\bf y}}
\newcommand{\CC}{{\mathbb{C}}}
\newcommand{\ZZ}{{\mathbb{Z}}}


\title{Introduction to Cluster Algebras}

\begin{document}
\begin{abstract}
  These are notes for a series of lectures presented at the ASIDE conference 2016.
\end{abstract}
\maketitle

%%%%%%%%%%%%%%%%%%%%%%
\section{Introduction}
  Cluster algebras were introduced by Fomin and Zelevinsky \cite{FZ02} in 2002 as the culmination of their study of total positivity \cite{FZ99} and (dual) canonical bases.  The topic of cluster algebras quickly grew into its own as a subject deserving independent study mainly fueled by its close relationship to many areas of mathematics.  Here is a partial list of related topics: combinatorics \cite{musiker-propp}, hyperbolic geometry \cite{FG,FST,MSW}, Lie theory \cite{GLS}, Poisson geometry \cite{GSV}, integrable systems \cite{dFK,G}, representations of associative algebras \cite{CC,CK,BMRRT,Rup1,Q,Rup2}, mathematical physics \cite{EF,ABCGPT}, and quantum groups \cite{K,GLS2,KQ,BR}.

  In these notes we will give an introduction to cluster algebras and a couple of the applications mentioned above.  These notes are far from exhaustive and the references above only touch on the vast literature.  Other overviews of cluster algebras can be found in the works \cite{??} which will also provide additional references.

  The paper is organized as follows.  Section~\ref{sec:motivation} gives several motivating examples from which we will abstract the definition of a cluster algebra.  The reader will be guided towards providing ad-hoc proofs of some of the important theorems of cluster algebra theory in the exercises of that section.  Section~\ref{sec:cluster_algebras} contains several variations on the definition of cluster algebras with increasing general notions of coefficients culminating in cluster algebras defined over an arbitrary semifield.  In Section~\ref{sec:basic_theorems} we describe the foundational results in the theory of cluster algebras and sketch or otherwise indicate the ideas behind the proofs of these results. Section~\ref{sec:poisson_and_quantum} recalls the theory of Poisson structures compatible with a cluster algebra and describes how this naturally leads to a quantization of cluster algebras.  Finally we conlclude with applications of the cluster algebra machinery to problems involving integrable systems in Section~\ref{sec:integrable_systems}.


%%%%%%%%%%%%%%%%%%%%%%%%%%%%%
\section{Motivating Examples}\label{sec:motivation}
  <Write brief lead-in.>

  \begin{example}
    A Markov triple is a tuple $(a,b,c)$ of positive integers satisfying the Markov equation $a^2+b^2+c^2=3abc$, an integer which appears as a term in a Markov triple is called a Markov number.  The Markov equation is an example of a Diophantine equation and two classical number theoretic problems are to determine the number of solutions and to determine a method for finding all such solutions.  We will solve both of these problems for the Markov equation.

    To begin with we note that $(1,1,1)$ and $(1,1,2)$ are (up to reordering) the only Markov triples with repeated values.
    \begin{subexercise}
      Prove this claim.
    \end{subexercise}
    Rearranging the Markov equation we see that $c^2-3abc+a^2+b^2=0$ and so $c$ is a root of the quadratic $f(x)=x^2-3abx+a^2+b^2$.  But the other root $c'=3ab-c=\frac{a^2+b^2}{c}$ is a positive integer and so $(a,b,c')$ is again a Markov triple.  

    Note that there was nothing special about $c$ in the calculation above so that, given any Markov triple $(a,b,c)$ we may perform three possible exchanges 
    \[(a,b,3ab-c)\quad(a,3ac-b,c)\quad(3bc-a,b,c)\]
    and obtain another Markov triple in each case.  The following exercises solve the above two classical problems of Diophantine equations.
    \begin{subexercise}\mbox{}
      \begin{enumerate}[\quad\upshape (a)]
        \item Prove that there are infinitely many Markov triples by showing that there is no bound on how large the largest value can be.
        \item Show that all Markov triples may be obtained from the Markov triple $(1,1,1)$ by a sequence of exchanges.
      \end{enumerate}
    \end{subexercise}
  \end{example}

  \begin{example}
    $Gr_{2}(\CC^n)$
  \end{example}
  
  \begin{example}
    Total positivity in $SL_n$.  <Talk about double Bruhat cells?>
  \end{example}

  \begin{example}
    <Include some example where $y$-variable mutations are natural.>
  \end{example}

  \begin{exercise}
    Anything illuminating to ask participants to do?
  \end{exercise}

%%%%%%%%%%%%%%%%%%%%%%%%%%%%%%%%%%%%%%%%%%%%%%
\section{A Unifying Concept: Cluster Algebras}\label{sec:cluster_algebras}
	%We define (skew-symmetric) cluster algebras at three successive levels of generality, differing from one another in the allowable types of coefficients. 
	We define (skew-symmetric) cluster algebras of geometric type and $Y$-patterns, with a focus on the underlying dynamics of seed mutations.
	
	\subsection{Trivial coefficients}

	\begin{definition}
		A \emph{seed} is a pair $(\x,B)$ where $\x = (x_1,\ldots, x_n)$ is an $n$-tuple of elements of a rational function field and $B$ is a skew-symmetric integer $n \times n$ matrix.  The vector $\x$ is called the \emph{cluster} and the matrix $B$ is called the \emph{exchange matrix}.
	\end{definition}
  
	The following employs the notation $[a]_+ = \max(a,0)$.
	\begin{definition}
		Given a seed ($\x,B$) and an integer $k=1,2,\ldots, n$ the \emph{seed mutation} $\mu_k$ in direction $k$ produces a new seed $\mu_k(\x,B) = (\x',B')$ where $\x' = (x_1,\ldots, x_{k-1}, x_k', x_{k+1},\ldots, x_n)$ with 
		\begin{equation} \label{eq:exchange relation}
		x_k' = \frac{\prod_{b_{ik}>0} x_i^{b_{ik}} + \prod_{b_{ik}<0} x_i^{-b_{ik}}}{x_k}
		\end{equation}
		and $B'$ is defined by
		\begin{equation}\label{eq:matrix mutation}
      b'_{ij}=\begin{cases}
                 -b_{ij} & \text{if $i=k$ or $j=k$;}\\
                 b_{ij}+[b_{ik}]_+[b_{kj}]_+-[-b_{ik}]_+[-b_{kj}]_+ & \text{otherwise.}
               \end{cases}
    \end{equation}
	\end{definition}
		
	In words, the mutation $\mu_k$ has the following effects:
	\begin{enumerate}
	\item $x_k$ changes to $x_k'$ satisfying $x_kx_k' = (\textrm{ binomial in the other } x_i$),
	\item entries $b_{ij}$ of $B$ away from row and column $k$ increase (resp. decrease) by $b_{ik}b_{kj}$ if $b_{ik}$ and $b_{kj}$ are both positive (resp. negative),
	\item row and column $k$ of $B$ are negated.
	\end{enumerate}
	
	\begin{definition}
		Fix an ambient field $\cF = \CC(x_1,\ldots, x_n)$ and an initial seed $((x_1,\ldots, x_n), B)$.  The entries of the clusters of all seeds reachable from this one by a sequence of mutations are called the \emph{cluster variable}.  The \emph{cluster algebra} associated with the initial seed is the subalgebra $\cA:=\cA(\bx,B)$ of $F$ generated by the set of all cluster variables.
	\end{definition}
	
	
  \begin{example}
    Type $A_2$.
  \end{example}
	
	\begin{theorem}
	Laurent phenomenon (minimal generality)
	\end{theorem}
	
	\subsection{Geometric type}
	Geometric type cluster algebras generalize the previous setting by allowing clusters to contain so-called frozen variables in addition to cluster variables.  The frozen variables do not mutate themselves, but they can take part in the exchange relations for cluster variables.  An \emph{extended cluster}, by convention, is typically written $\x = (x_1,\ldots, x_n, x_{n+1},\ldots x_m)$ where $x_1,\ldots, x_n$ are the cluster variable and $x_{n+1},\ldots, x_m$ are the frozen variables.  An \emph{extended exchange matrix} is an $m \times n$ integers matrix $\tilde{B}$ with the property that its top $n \times n$ part is skew-symmetric.
	
	\begin{definition}
	Fix a seed $(\x, \tilde B)$ with $\x=(x_1,\ldots, x_m)$ an extended cluster and $\tilde{B}$ an extended exchange matrix.  For an integer $k = 1,2,\ldots, n$, the \emph{seed mutation} $\mu_k$ in direction $k$ produces a new seed $\mu_k(\x,\tilde{B}) = (\x',\tilde{B}')$ with $\x' = (x_1,\ldots, x_{k-1}, x_k', x_{k+1},\ldots, x_m)$.  The formulas for $x_k'$ and the entries $\tilde{b}'_{ij}$ of $B'$ are the same as in \eqref{eq:exchange relation} and \eqref{eq:matrix mutation}, where in the former that the products now range from $1$ to $m$ instead of from $1$ to $n$.
	\end{definition}
	
	In this setting, the cluster algebra is the subalgebra of the ambient field generated by all cluster variables reachable from a given seed, together with the frozen variables which are constant across all seeds.
	
	There is an alternate formulation of cluster algebras of geometric type which focuses on the roles the frozen variables $x_{n+1},\ldots, x_m$ play in the exchange relations, rather than on the variables themselves.  Let $(\x, \tilde{B})$ be an extended seed.  For $k=1,\ldots, n$ let 
	\begin{displaymath}
	y_k = \prod_{i=n+1}^m x_i^{b_{ik}}.
	\end{displaymath}
	Define an operation (called auxiliary addition) $\oplus$ on Laurent monomials by
	\begin{displaymath}
	\prod_{i=n+1}^m x_i^{e_i} \oplus \prod_{i=n+1}^m x_i^{f_i} = \prod_{i=n+1}^m x_i^{\min(e_i,f_i)}.
	\end{displaymath}
	Using this operation, we can extract positive and negative exponents as follows
	\begin{displaymath}
	\frac{y_k}{1 \oplus y_k} = \prod_{\substack{i=n+1,\ldots, m\\ b_{ik > 0}}} x_i^{b_{ik}}
	\quad \quad \quad \frac{1}{1 \oplus y_k} = \prod_{\substack{i=n+1,\ldots, m\\ b_{ik < 0}}} x_i^{-b_{ik}}.
	\end{displaymath}
	The change in going from the original exchange relation \eqref{eq:exchange relation} to the one in geometric type, then, can be summarized by saying that the two terms of the binomial are enriched with coefficients $y_k/(1 \oplus y_k)$ and $1/(1 \oplus y_k)$.  We now write
	\begin{displaymath} 
		x_k' = \frac{y_k\prod_{b_{ik}>0} x_i^{b_{ik}} + \prod_{b_{ik}<0} x_i^{-b_{ik}}}{(1 \oplus y_k)x_k}
	\end{displaymath}
	where the products are for $i$ from $1$ to $n$.
	
	\subsection{$Y$-patterns}
	There is an alternate version of seeds and mutations, closely related to the previous, which itself arises in many applications.  
	
	\begin{definition}
		A \emph{$Y$-seed} is a pair $(\y,B)$ consisting of an $n$-tuple $\y=(y_1,\ldots, y_n)$ of rational functions and an $n\times n$ skew symmetric matrix $B$.  For $k=1,\ldots, n$, the \emph{$Y$-seed mutation} $\mu_k$ is defined by
		\begin{displaymath}
			\mu_k((y_1,\ldots, y_n),B) = ((y_1',\ldots, y_n'),B')
		\end{displaymath}
		for $B'$ as defined in \eqref{eq:matrix mutation} and 
		\begin{equation}\label{eq:y mutation}
		% Warning: this seems to only agree with (2.3) in CA-IV in the skew-symmetric case
		y_j' = \begin{cases}
		y_k^{-1} & \text{if $j = k$;} \\
		y_jy_k^{|-b_{jk}|_+}(1+y_k)^{b_{jk}} & \text{if $j \neq k$.} \\
		\end{cases}
		\end{equation}
	\end{definition}
	
	In words, the $Y$-seed mutation $\mu_k$ has the following effects:
	\begin{enumerate}
	\item For each $j \neq k$, $y_j$ is multiplied by $(1+y_k)^{b_{jk}}$ if $b_{jk}>0$ and by $(1+y_k^{-1})^{b_{jk}}$ if $b_{jk}<0$,
	\item $y_k$ is inverted,
	\item $B$ is changed in the same way as for ordinary seed mutations.
	\end{enumerate}
	%The $Y$-dynamics also go by the name \emph{coefficient dynamics} because the tropicalisation of \cite{eq:y mutation} describes how the coefficients in a geometric type cluster algebra evolve.
	
	One connection between the two types of dynamics comes by way of a certain Laurent monomial change of variables.  Let $\tilde{B}$ be an $m \times n$, extended exchange matrix with $B$ its upper $n \times n$ part.  Given a seed $((x_1,\ldots, x_m),\tilde{B})$, define an associated $Y$-seed $((\hat{y}_1,\ldots, \hat{y}_n),B)$ by
	\begin{displaymath}
	\hat{y}_j = \prod_{i=1}^m x_i^{b_{ij}}
	\end{displaymath}
	
	\begin{proposition}
	Fix $k=1,\ldots, n$ and suppose $\mu_k(\x,\tilde{B}) = (\x',\tilde{B}')$.  Define $(y_1,\ldots, y_n)$ from $(\x,\tilde{B})$ and $(y_1',\ldots, y_n')$ from $(\x',\tilde{B}')$ as above.  Then 
	\begin{displaymath}
	\mu_k((\hat{y}_1,\ldots, \hat{y}_n), B) = ((\hat{y}_1',\ldots, \hat{y}_n'), B').
	\end{displaymath}
	\end{proposition}

  \begin{exercise}
    Compute all cluster variables and coefficient variables for cluster algebras associated to $B=\left[\begin{array}{cc} 0 & b\\ -c & 0\end{array}\right]$ with $b,c\in\ZZ_{>0}$ and $bc\le 3$.  Justify why attempting such a calculation is futile for $bc\ge4$.
  \end{exercise}

  \begin{exercise}
    Prove the Laurent phenomenon for rank 2 cluster algebras.  <Include guiding hints and make the question more explicit.>
  \end{exercise}

%%%%%%%%%%%%%%%%%%%%%%%
\section{Basic Results}\label{sec:basic_theorems}

  \begin{theorem}
    Laurent phenomenon.
  \end{theorem}

  %Thus it is natural to seek a concrete understanding of these Laurent expansions of cluster variables.
  %\begin{theorem}
  %  Categorification?
  %\end{theorem}

  \begin{theorem}
    Finite-type classification.
  \end{theorem}

  \begin{theorem}
    Positivity.
  \end{theorem}

  %The proof builds on a concrete combinatorial construction of cluster variables in rank 2.
  %\begin{theorem}
  %  Combinatorial constructions in rank 2.
  %\end{theorem}

%%%%%%%%%%%%%%%%%%%%%%%%%%%%%%%%%%%%%%%%%%%%%%%%%%%%%%%%
\section{Compatible Poisson Structures and Quantization}\label{sec:poisson_and_quantum}

  \begin{definition}
    A Poisson algebra $(A,\{\cdot,\cdot\})$ is an associative algebra $A$ equipped with an additional skew-symmetric binary operation (written as $\{x,y\}$ for $x,y\in A$) called the Poisson bracket for which the following holds:
    \begin{itemize}
      \item given any $x\in A$, the automorphism $\{x,\cdot\}:A\to A$, $y\mapsto\{x,y\}$ is a derivation with respect to both binary operations on $A$, i.e. for $x,y,z\in A$ we have
      \begin{equation}
        \{x,yz\}=\{x,y\}z+y\{x,z\}\quad\text{and}\quad\{x,\{y,z\}\}=\{\{x,y\},z\}+\{y,\{x,z\}\}.
      \end{equation}
    \end{itemize}
  \end{definition}

  \begin{example}
    A symplectic manifold $(M,\omega)$ is a smooth even-dimensional manifold $M^{2n}$ together with a non-degenerate 2-form $\omega\in H^2(M)$, i.e. $\omega^n\ne0$ is a volume form.  The symplectic structure $\omega$ provides a natural association of a vector field $\xi_f$ to a function $f\in\cO(M)$ via the contraction formula $\iota_{\xi_f}\omega=-df$.  The algebra of smooth functions on $M$ is then naturally a Poisson algebra via $\{f,g\}=\omega(\xi_f,\xi_g)$ for $f,g\in\cO(M)$.
  \end{example}
  More generally, any smooth manifold together with a Poisson structure on its algebra of regular functions is called a Poisson manifold.

  \begin{definition}
    Let $A$ be a Poisson algebra.  A finite collection $\bff\{f_i\}\subset A$ is called log-canonical if there exists a matrix $\Omega$ so that $\{f_i,f_j\}=\Omega_{ij}f_if_j$ for $f_i,f_j\in\bff$.
  \end{definition}
  This terminology is motivated by the property $\{\log f_i,\log f_j\}=\Omega_{ij}$ for $f_i,f_j\in\bff$.

  \begin{definition}
    Let $\cA$ be a cluster algebra.  A Poisson bracket $\{\cdot,\cdot\}$ on $\cA$ is compatible with the cluster algebra structure every cluster $\bx$ of $\cA$ is log-canonical with respect to $\{\cdot,\cdot\}$.
  \end{definition}

  \begin{lemma}
  \label{le:compatibility}
    Let $\cA(\bx,B)$ be a cluster algebra with compatible Poisson structure $\{\cdot,\cdot\}$.  Then writing $\Omega$ for the compatibility matrix, we have $B^T\Omega=[D\, \boldsymbol{0}]$ where $D$ is a diagonal matrix which skew-symmetrizes $B$.
  \end{lemma}
  \begin{exercise}
    Prove Lemma~\ref{le:compatibility}. 
  \end{exercise}

  \begin{theorem}
    A compatible Poisson structure exists if and only if the extended exchange matrix $\tilde B$ has full rank.  Moreover, the collection of all such Poisson structures is parametrized by an affine space of dimension $r(B)+{m\choose 2}$.
  \end{theorem}

  Given a cluster algebra $\cA$ of full rank, any choice of compatible Poisson structure gives rise to a quantization of $\cA$ as follows.

  \begin{definition}
    Quantum cluster algebra.
  \end{definition}

%%%%%%%%%%%%%%%%%%%%%%%%%%%%%%%%%%%%%%%%%%%%
\section{Applications to Integrable Systems}\label{sec:integrable_systems}


%%%%%%%%%%%%%%%%%%%%%%%%%%%
\begin{thebibliography}{99}
  \bibitem[FZ02] S. Fomin and A. Zelevinsky, Cluster algebras I: Foundations
\end{thebibliography}

\end{document}