\documentclass{amsart}
\usepackage{amsmath,amssymb,latexsym}


\newtheorem{theorem}{Theorem}[section]
\newtheorem{conjecture}[theorem]{Conjecture}
\newtheorem{corollary}[theorem]{Corollary}
\newtheorem{definition}[theorem]{Definition}
\newtheorem{lemma}[theorem]{Lemma}
\newtheorem{proposition}[theorem]{Proposition}
\newtheorem{example}[theorem]{Example}
\newtheorem{exercise}[theorem]{Exercise}
\theoremstyle{remark} 
\newtheorem{remark}[theorem]{Remark}
\numberwithin{equation}{section}


\newcommand{\cA}{\mathcal{A}}
\newcommand{\cF}{\mathcal{F}}
\newcommand{\x}{\textbf{x}}
\newcommand{\CC}{\mathbb{C}}
\newcommand{\ZZ}{\mathbb{Z}}


\title{Introduction to Cluster Algebras}

\begin{document}
\begin{abstract}
  These are notes for a series of lectures presented at the ASIDE conference 2016.
\end{abstract}
\maketitle

%%%%%%%%%%%%%%%%%%%%%%
\section{Introduction}
  <Write overview of cluster algebras and broad history.>

%%%%%%%%%%%%%%%%%%%%%%%%%%%%%
\section{Motivating Examples}
  <Write brief lead-in.>

  \begin{example}
    Markov numbers. Ad hoc proofs of: integrality and positivity.
  \end{example}

  \begin{example}
    $Gr_{2}(\CC^n)$
  \end{example}
  
  \begin{example}
    Total positivity in $SL_n$.  <Talk about double Bruhat cells?>
  \end{example}

  \begin{example}
    <Include some example where $y$-variable mutations are natural.>
  \end{example}

  \begin{exercise}
    Anything illuminating to ask participants to do?
  \end{exercise}

%%%%%%%%%%%%%%%%%%%%%%%%%%%%%%%%%%%%%%%%%%%%%%
\section{A Unifying Concept: Cluster Algebras}
	We define (skew-symmetric) cluster algebras at three successive levels of generality, differing from one another in the allowable types of coefficients.
	
	\subsection{Trivial coefficients}

	\begin{definition}
		A \emph{seed} is a pair $(\x,B)$ where $\x = (x_1,\ldots, x_n)$ is an $n$-tuple of elements of a rational function field and $B$ is a skew-symmetric integer $n \times n$ matrix.  The vector $\x$ is called the \emph{cluster} and the matrix $B$ is called the \emph{exchange matrix}.
	\end{definition}
  
	The following employs the notation $[a]_+ = \max(a,0)$.
	\begin{definition}
		Given a seed ($\x,B$) and an integer $k=1,2,\ldots, n$ the \emph{seed mutation} $\mu_k$ in direction $k$ produces a new seed $\mu_k(\x,B) = (\x',B')$ where $\x' = (x_1,\ldots, x_{k-1}, x_k', x_{k+1},\ldots, x_n)$ with 
		\begin{equation} \label{eq:exchange relation}
		x_k' = \frac{\prod_{b_{ik}>0} x_i^{b_{ik}} + \prod_{b_{ik}<0} x_i^{-b_{ik}}}{x_k}
		\end{equation}
		and $B'$ is defined by
		\begin{equation}\label{eq:matrix mutation}
      b'_{ij}=\begin{cases}
                 -b_{ij} & \text{if $i=k$ or $j=k$;}\\
                 b_{ij}+[b_{ik}]_+b_{kj}+b_{ik}[-b_{kj}]_+ & \text{otherwise.}
               \end{cases}
    \end{equation}
	\end{definition}
		
	In words, the mutation $\mu_k$ has the following effects:
	\begin{enumerate}
	\item $x_k$ changes to $x_k'$ satisfying $x_kx_k' = (\textrm{ binomial in the other } x_i$),
	\item entries $b_{ij}$ of $B$ away from row and column $k$ increase (resp. decrease) by $b_{ik}b_{kj}$ if $b_{ik}$ and $b_{kj}$ are both positive (resp. negative),
	\item row and column $k$ of $B$ are negated.
	\end{enumerate}
	
	\begin{definition}
		Fix an ambient field $\cF = \CC(x_1,\ldots, x_n)$ and an initial seed $((x_1,\ldots, x_n), B)$.  The entries of the clusters of all seeds reachable from this one by a sequence of mutations are called the \emph{cluster variable}.  The \emph{cluster algebra} associated with the initial seed is the subalgebra $\cA$ of $F$ generated by the set of all cluster variables.
	\end{definition}
	
	
  \begin{example}
    Type $A_2$.
  \end{example}
	
	\begin{theorem}
	Laurent phenomenon (minimal generality)
	\end{theorem}
	
	\subsection{Geometric type}
	Geometric type cluster algebras generalize the previous setting by allowing clusters to contain so-called frozen variables in addition to cluster variables.  The frozen variables do not mutate themselves, but they can take part in the exchange relations for cluster variables.  An \emph{extended cluster}, by convention, is typically written $\x = (x_1,\ldots, x_n, x_{n+1},\ldots x_m)$ where $x_1,\ldots, x_n$ are the cluster variable and $x_{n+1},\ldots, x_m$ are the frozen variables.  An \emph{extended exchange matrix} is an $m \times n$ integers matrix $\tilde{B}$ with the property that its top $n \times n$ part is skew-symmetric.
	
	\begin{definition}
	Fix a seed $(\x, B)$ with $\x=(x_1,\ldots, x_m)$ an extended cluster and $\tilde{B}$ an extended exchange matrix.  For an integer $k = 1,2,\ldots, n$, the \emph{seed mutation} $\mu_k$ in direction $k$ produces a new seed $\mu_k(\x,\tilde{B}) = (\x',\tilde{B}')$ with $\x' = (x_1,\ldots, x_{k-1}, x_k', x_{k+1},\ldots, x_m)$.  The formulas for $x_k'$ and the entries $\tilde{b}'_{ij}$ of $B'$ are the same as in \eqref{eq:exchange relation} and \eqref{eq:matrix mutation}, where in the former that the products now range from $1$ to $m$ instead of from $1$ to $n$.
	\end{definition}
	
	In this setting, the cluster algebra is the subalgebra of the ambient field generated by all cluster variables reachable from a given seed, together with the frozen variables which are constant across all seeds.
	
	There is an alternate formulation of cluster algebras of geometric type which focuses on the roles the frozen variables $x_{n+1},\ldots, x_m$ play in the exchange relations, rather than on the variables themselves.  Let $(\x, \tilde{B})$ be an extended seed.  For $k=1,\ldots, n$ let 
	\begin{displaymath}
	y_k = \prod_{i=n+1}^m x_i^{b_{ik}}.
	\end{displaymath}
	Define an operation (called auxiliary addition) $\oplus$ on Laurent monomials by
	\begin{displaymath}
	\prod_{i=n+1}^m x_i^{e_i} \oplus \prod_{i=n+1}^m x_i^{f_i} = \prod_{i=n+1}^m x_i^{\min(e_i,f_i)}.
	\end{displaymath}
	Using this operation, we can extract positive and negative exponents as follows
	\begin{displaymath}
	\frac{y_k}{1 \oplus y_k} = \prod_{\substack{i=n+1,\ldots, m\\ b_{ik > 0}}} x_i^{b_{ik}}
	\quad \quad \quad \frac{1}{1 \oplus y_k} = \prod_{\substack{i=n+1,\ldots, m\\ b_{ik < 0}}} x_i^{-b_{ik}}.
	\end{displaymath}
	The change in going from the original exchange relation \eqref{eq:exchange relation} to the one in geometric type, then, can be summarized by saying that the two terms of the binomial are enriched with coefficients $y_k/(1 \oplus y_k)$ and $1/(1 \oplus y_k)$.  We now write
	\begin{displaymath} 
		x_k' = \frac{y_k\prod_{b_{ik}>0} x_i^{b_{ik}} + \prod_{b_{ik}<0} x_i^{-b_{ik}}}{(1 \oplus y_k)x_k}
	\end{displaymath}
	where the products are for $i$ from $1$ to $n$.
	
	\subsection{General coefficients}

  \begin{exercise}
    Compute all cluster variables and coefficient variables for cluster algebras associated to $B=\left[\begin{array}{cc} 0 & b\\ -c & 0\end{array}\right]$ with $b,c\in\ZZ_{>0}$ and $bc\le 3$.  Justify why attempting such a calculation is futile for $bc\ge4$.
  \end{exercise}

  \begin{exercise}
    Prove the Laurent phenomenon for rank 2 cluster algebras.  <Include guiding hints and make the question more explicit.>
  \end{exercise}

%%%%%%%%%%%%%%%%%%%%%%%
\section{Basic Results}

  \begin{theorem}
    Laurent phenomenon.
  \end{theorem}

  %Thus it is natural to seek a concrete understanding of these Laurent expansions of cluster variables.
  %\begin{theorem}
  %  Categorification?
  %\end{theorem}

  \begin{theorem}
    Finite-type classification.
  \end{theorem}

  \begin{theorem}
    Positivity.
  \end{theorem}

  %The proof builds on a concrete combinatorial construction of cluster variables in rank 2.
  %\begin{theorem}
  %  Combinatorial constructions in rank 2.
  %\end{theorem}

%%%%%%%%%%%%%%%%%%%%%%%%%%%%%%%%%%%%%%%%%%%%%%%%%%%%%%%%
\section{Compatible Poisson Structures and Quantization}

  \begin{definition}
    Poisson algebra.
  \end{definition}

  \begin{definition}
    Log-canonical Poisson brackets.
  \end{definition}

  \begin{definition}
    Compatible Poisson structures on a cluster algebra.
  \end{definition}

  \begin{theorem}
    A compatible Poisson structure exists if and only if the exchange matrix has full rank.  Moreover, the collection of all such Poisson structures is parametrized by an affine space of dimension ???.
  \end{theorem}

  \begin{definition}
    Quantum cluster algebra.
  \end{definition}

%%%%%%%%%%%%%%%%%%%%%%%%%%%%%%%%%%%%%%%%%%%%
\section{Applications to Integrable Systems}


%%%%%%%%%%%%%%%%%%%%%%%%%%%
\begin{thebibliography}{99}
  \bibitem[FZ02] S. Fomin and A. Zelevinsky, Cluster algebras I: Foundations
\end{thebibliography}

\end{document}