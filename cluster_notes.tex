\documentclass{amsart}
\usepackage{amsmath,amssymb,latexsym,enumerate}
\usepackage{pstricks}
\usepackage{pst-node}


\newtheorem{theorem}{Theorem}[section]
\newtheorem{conjecture}[theorem]{Conjecture}
\newtheorem{corollary}[theorem]{Corollary}
\newtheorem{lemma}[theorem]{Lemma}
\newtheorem{proposition}[theorem]{Proposition}
\theoremstyle{definition}
\newtheorem{definition}[theorem]{Definition}
\newtheorem{example}[theorem]{Example}
\newtheorem{subexercise}{Exercise}[theorem]
\newtheorem{exercise}[theorem]{Exercise}
\theoremstyle{remark} 
\newtheorem{remark}[theorem]{Remark}
\numberwithin{equation}{section}


\newcommand{\cA}{{\mathcal{A}}}
\newcommand{\cF}{{\mathcal{F}}}
\newcommand{\cL}{{\mathcal{L}}}
\newcommand{\cO}{{\mathcal{O}}}
\newcommand{\cU}{{\mathcal{U}}}
\newcommand{\bff}{{\bf f}}
\newcommand{\x}{{\bf x}}
\newcommand{\bx}{{\bf x}}
\newcommand{\y}{{\bf y}}
\newcommand{\by}{{\bf y}}
\newcommand{\CC}{{\mathbb{C}}}
\newcommand{\ZZ}{{\mathbb{Z}}}

\newcommand{\meet}[2]{#1 \cap #2}
%\newcommand{\join}[2]{\overleftrightarrow{#1 #2}}
\newcommand{\join}[2]{\langle #1, #2 \rangle}

\newcommand{\erase}[1]{{}}


\title{Introduction to Cluster Algebras}

\begin{document}
\begin{abstract}
  These are notes for a series of lectures presented at the ASIDE conference 2016.
\end{abstract}
\maketitle

%%%%%%%%%%%%%%%%%%%%%%
\section{Introduction}
  Cluster algebras were introduced by Fomin and Zelevinsky \cite{FZ02} in 2002 as the culmination of their study of total positivity \cite{FZ00} and (dual) canonical bases.  The topic of cluster algebras quickly grew into its own as a subject deserving independent study mainly fueled by its close relationship to many areas of mathematics.  Here is a partial list of related topics: combinatorics \cite{MP07}, hyperbolic geometry \cite{FG06,FST08,MSW13}, Lie theory \cite{GLS06}, Poisson geometry \cite{GSV03}, integrable systems \cite{dFK10,G11}, representations of associative algebras \cite{CC,CK,BMRRT,Rup1,Q,Rup2}, mathematical physics \cite{EF,ABCGPT}, and quantum groups \cite{K,GLS2,KQ,BR}.

  In these notes we will give an introduction to cluster algebras and a couple of the applications mentioned above.  These notes are far from exhaustive and the references above only touch on the vast literature.  Other overviews of cluster algebras can be found in the works \cite{??} which will also provide additional references.

  The paper is organized as follows.  Section~\ref{sec:motivation} gives several motivating examples from which we will abstract the definition of a cluster algebra.  The reader will be guided towards providing ad-hoc proofs of some of the important theorems of cluster algebra theory in the exercises of that section.  Section~\ref{sec:cluster_algebras} contains several variations on the definition of cluster algebras with increasing general notions of coefficients culminating in cluster algebras defined over an arbitrary semifield.  In Section~\ref{sec:basic_theorems} we describe the foundational results in the theory of cluster algebras and sketch or otherwise indicate the ideas behind the proofs of these results. Section~\ref{sec:poisson_and_quantum} recalls the theory of Poisson structures compatible with a cluster algebra and describes how this naturally leads to a quantization of cluster algebras.  Finally we conlclude with applications of the cluster algebra machinery to problems involving integrable systems in Section~\ref{sec:integrable_systems}.


%%%%%%%%%%%%%%%%%%%%%%%%%%%%%
\section{Motivating Examples}\label{sec:motivation}
  <Write brief lead-in.>

  \begin{example}
    A Markov triple is a tuple $(a,b,c)$ of positive integers satisfying the Markov equation $a^2+b^2+c^2=3abc$, an integer which appears as a term in a Markov triple is called a Markov number.  The Markov equation is an example of a Diophantine equation and two classical number theoretic problems are to determine the number of solutions and to determine a method for finding all such solutions.  We will solve both of these problems for the Markov equation.

    To begin with we note that $(1,1,1)$ and $(1,1,2)$ are (up to reordering) the only Markov triples with repeated values.
    \begin{subexercise}
      Prove this claim.
    \end{subexercise}
    Rearranging the Markov equation we see that $c^2-3abc+a^2+b^2=0$ and so $c$ is a root of the quadratic $f(x)=x^2-3abx+a^2+b^2$.  But the other root $c'=3ab-c=\frac{a^2+b^2}{c}$ is a positive integer and so $(a,b,c')$ is again a Markov triple.  

    Note that there was nothing special about $c$ in the calculation above so that, given any Markov triple $(a,b,c)$ we may perform three possible exchanges 
    \[(a,b,3ab-c)\quad(a,3ac-b,c)\quad(3bc-a,b,c)\]
    and obtain another Markov triple in each case.  The following exercises solve the above two classical problems of Diophantine equations.
    \begin{subexercise}\mbox{}
      \begin{enumerate}[\quad\upshape (a)]
        \item Prove that there are infinitely many Markov triples by showing that there is no bound on how large the largest value can be.
        \item Show that all Markov triples may be obtained from the Markov triple $(1,1,1)$ by a sequence of exchanges.
      \end{enumerate}
    \end{subexercise}
  \end{example}

  \begin{example} \label{example:Gr2n}
		The \emph{Grassmannian} $Gr_{k}(\CC^n)$ is the set of $k$-dimensional linear subspaces of $\CC^n$.  A point in the Grassmannian can be described, albeit non-uniquely, as the row span of a full rank matrix $A \in \CC^{k \times n}$.  The maximal minors of $A$ are called \emph{Pl\"ucker coordinates}.
		
		Now, restrict to $k=2$ and let 
		\begin{displaymath}
			A = \left[ \begin{array}{cccc}
			a_{11} & a_{12} & \ldots & a_{1n} \\
			a_{21} & a_{22} & \ldots & a_{2n} \\
			\end{array} \right].
		\end{displaymath}
		The Pl\"ucker coordinates are given by $\Delta_{ij} = a_{1i}a_{2j} - a_{1j}a_{2i}$ for $1 \leq i < j \leq n$.  
		
		\begin{subexercise}
			The $\Delta_{ij}$ satisfy the so-called \emph{Pl\"ucker relations}
			\begin{displaymath}
				\Delta_{ik}\Delta_{jl} = \Delta_{ij}\Delta_{kl} + \Delta_{il}\Delta_{jk}
			\end{displaymath}
			for $1 \leq i < j < k < l \leq n$.
		\end{subexercise}
		
		Consider a regular $n$-gon with vertices labeled $1,2,\ldots, n$.  Let $T$ be a triangulation, i.e. a maximal collection of chords $\overline{ij}$ with $1 \leq i < j \leq n$, no two of which intersect in their interior.  Note that $T$ always consists of $n-3$ diagonals together with the $n$ sides $\overline{12}, \overline{23}, \overline{34}, \ldots, \overline{1n}$.  Associated to $T$ is a corresponding collection of Pl\"ucker coordinates
		\begin{displaymath}
			\x_T := \{\Delta_{ij} : \overline{ij} \in T\}
		\end{displaymath}
		\begin{proposition}
		Fix positive reals $x_{ij}$ for all $\overline{ij} \in T$.  Then there exists $A \in Gr_2(\CC^n)$ such that $\Delta_{ij}(A) = x_{ij}$ for all $\overline{ij} \in T$.  Moreover, each $\Delta_{ij}(A)$ with $\overline{ij} \notin T$ is uniquely determined by the $x_{ij}$ via a subtraction-free rational expression.
		\end{proposition}
		
		For instance, if $n=4$ and $T = \{\overline{13},\overline{12},\overline{23},\overline{34},\overline{14}\}$ then the Pl\"ucker relation implies
		\begin{displaymath}
		\Delta_{24}(A) = \frac{x_{12}x_{34} + x_{14}x_{23}}{x_{13}}.
		\end{displaymath}
		This formula can be thought of as a change of coordinate systems from the original one to one corresponding to $T' = (T \setminus \{\overline{13}\}) \cup \{\overline{24}\}$.  More generally, if $n>4$ then the rational expressions promised by the Proposition can be obtained by performing a sequence of such flips on various sub-quadrilaterals of the $n$-gon with vertices $i,j,k,l$.
    \begin{exercise}
      The change of coordinate systems observed above can be understood more classically in terms of the Ptolemy relations for cyclic quadrilaterals.  Indeed, consider four distinct points labelled $1,2,3,4$ on a circle and let $x_{ij}$ denote the distance between vertices $i$ and $j$.  Prove that $x_{13}x_{24}=x_{12}x_{34}+x_{14}x_{23}$.
    \end{exercise}
  \end{example}
  
  \begin{example}\label{example:total positivity}
    An $n\times n$ matrix $M$ is called \emph{totally positive} if the determinant of every square submatrix is a positive real number.  In particular, every entry of $M$ is positive and $M$ must be invertible.  Write $GL_n^{>0}\subset GL_n$ for the subset of totally positive matrices.  To check that a given matrix $M\in GL_n$ is totally positive one must, a priori, check that all ${2n\choose n}-1$ minors of $M$ are positive.  A natural question is whether this verification process can be made more efficient, i.e. is there a smaller collection of minors one may compute and from the positivity of this subset conclude that every minor is positive, i.e. conclude that $M\in GL_n^{>0}$?

    We will call such a collection a \emph{total positivity criterion}.  For small $n$ such criteria can be easily found and easily verified.  For example, a matrix $M=\left[\begin{array}{cc} a & b\\ c & d\end{array}\right]\in GL_2$ is totally positive if and only if $a,b,c,ad-bc>0$ if and only if $b,c,d,ad-bc>0$.
    \begin{exercise}
      Find a minimal collection of minors whose positivity guarantees a matrix in $GL_3$ is totally positive (hint: any such total positivity criterion consists of 9 minors).
    \end{exercise}

    To describe a solution and easily identify total positivity criteria for all groups $GL_n$, it will be convenient to slightly generalize the notion of total positivity.  An $n\times n$ matrix $M$ is called \emph{totally nonnegative} if the determinant of every square submatrix is a nonnegative real number.  Write $GL_n^{\ge0}\subset GL_n$ for the subset of totally nonnegative matrices.  Again one may ask: what is a minimal collection of minors needed to check that a matrix is totally nonnegative?  Unfortunately, or perhaps fortunately, the total nonnegativity criteria are not uniformly described across all of $GL_n$.  The solution to this problem naturally leads one to study certain subvarieties of $GL_n$ called \emph{double Bruhat cells}, which we now describe.

    Let $B_+,B_-\subset GL_n$ denote the subgroups of upper and lower triangular matrices respectively.  Identify the symmetric group $\Sigma_n$ with the subgroup of $GL_n$ consisting of permutation matrices, i.e. matrices having precisely one nonzero entry 1 in each row and column.  For example, identify the permutation $(12)\in\Sigma_2$ with the matrix $\left[\begin{array}{cc}0 & 1\\ 1 & 0\end{array}\right]$.  It is well known that $GL_n$ decomposes in two ways (actually many ways) as a union of \emph{Bruhat cells}:
    \[GL_n=\bigsqcup_{w\in\Sigma_n}B_+w B_+=\bigsqcup_{w\in\Sigma_n}B_-w B_-.\]
    To understand total nonnegativity criteria for $GL_n$, we will consider the \emph{double Bruhat cells} $G^{u,v}=B_+uB_+\cap B_-vB_-$.  It turns out that each double Bruhat cell admits its own total nonnegativity criteria, i.e. for each $u,v\in\Sigma_n$ there exists a minimal collection of minors whose positivity identifies the subset $G^{u,v}\cap GL_n^{\ge0}$ inside the double Bruhat cell $G^{u,v}$.



  \end{example}

  \begin{example}
    <Include some example where $y$-variable mutations are natural.>
  \end{example}

  \begin{exercise}
    Anything illuminating to ask participants to do?
  \end{exercise}

%%%%%%%%%%%%%%%%%%%%%%%%%%%%%%%%%%%%%%%%%%%%%%
\section{A Unifying Concept: Cluster Algebras}\label{sec:cluster_algebras}
	%We define (skew-symmetric) cluster algebras at three successive levels of generality, differing from one another in the allowable types of coefficients. 
	We define cluster algebras of geometric type and $Y$-patterns, with a focus on the underlying dynamics of seed mutations.
	
	\subsection{Basic definitions}

	\begin{definition}
		A \emph{seed} is a pair $(\x,B)$ where $\x = (x_1,\ldots, x_n)$ is an $n$-tuple of elements forming a transcendence basis of a rational function field and $B$ is a skew-symmetric integer $n \times n$ matrix.  The vector $\x$ is called the \emph{cluster} and the matrix $B$ is called the \emph{exchange matrix}.
	\end{definition}
  
	The following employs the notation $[a]_+ = \max(a,0)$.
	\begin{definition}
		Given a seed ($\x,B$) and an integer $k=1,2,\ldots, n$ the \emph{seed mutation} $\mu_k$ in direction $k$ produces a new seed $\mu_k(\x,B) = (\x',B')$ where $\x' = (x_1,\ldots, x_{k-1}, x_k', x_{k+1},\ldots, x_n)$ with 
		\begin{equation} \label{eq:exchange relation}
		x_k' = \frac{\prod_{b_{ik}>0} x_i^{b_{ik}} + \prod_{b_{ik}<0} x_i^{-b_{ik}}}{x_k}
		\end{equation}
		and $B'$ is defined by
		\begin{equation}\label{eq:matrix mutation}
      b'_{ij}=\begin{cases}
                 -b_{ij} & \text{if $i=k$ or $j=k$;}\\
                 b_{ij}+[b_{ik}]_+[b_{kj}]_+-[-b_{ik}]_+[-b_{kj}]_+ & \text{otherwise.}
               \end{cases}
    \end{equation}
	\end{definition}
		
	\begin{lemma} \label{lem:mutate}
	Let $(\x,B)$ be a seed and $k=1,\ldots, n$.
	\begin{enumerate}
		\item $\mu_k(\x,B)$ is also a seed
		\item $\mu_k(\mu_k(\x,B)) = (\x,B)$.
	\end{enumerate}
	\end{lemma}
	
	In words, the mutation $\mu_k$ has the following effects:
	\begin{enumerate}
	\item $x_k$ changes to $x_k'$ satisfying $x_kx_k' = (\textrm{ binomial in the other } x_i$),
	\item entries $b_{ij}$ of $B$ away from row and column $k$ increase (resp. decrease) by $b_{ik}b_{kj}$ if $b_{ik}$ and $b_{kj}$ are both positive (resp. negative),
	\item row and column $k$ of $B$ are negated.
	\end{enumerate}
	
	\begin{definition}
		Fix an ambient field $\cF = \CC(x_1,\ldots, x_n)$ and an initial seed $((x_1,\ldots, x_n), B)$.  The entries of the clusters of all seeds reachable from this one by a sequence of mutations are called the \emph{cluster variable}.  The \emph{cluster algebra} associated with the initial seed is the subalgebra $\cA:=\cA(\bx,B)$ of $F$ generated by the set of all cluster variables.
	\end{definition}
	
	
  \begin{example}
		Let $\x = (x_1,x_2)$ and
		\begin{displaymath}
			B = \left[ \begin{array}{cc}
			0 & 1 \\
			-1 & 0 \\
			\end{array}
			\right].
		\end{displaymath}
		Then $\mu_1(\x,B) = ((x_1',x_2), -B)$ where
		\begin{displaymath}
			x_1' = \frac{x_2+1}{x_1}.
		\end{displaymath}
		It is convenient to denote the new cluster variable $x_1'=x_3$.  Next $\mu_2((x_3,x_2),-B) = ((x_3,x_4),B)$ where
		\begin{displaymath}
			x_4 = \frac{x_3+1}{x_2} = \frac{\frac{x_2+1}{x_1}+1}{x_2} = \frac{x_1 + x_2 + 1}{x_1x_2}
		\end{displaymath}
		and  $\mu_2((x_3,x_4),B) = ((x_5,x_4),-B)$ where
		\begin{displaymath}
			x_5 = \frac{x_4+1}{x_3} = \frac{\frac{x_1 + x_2 + 1}{x_1x_2}+1}{\frac{x_2+1}{x_1}} = \frac{x_1+1}{x_2}.
		\end{displaymath}
		Remarkably, these are the only distinct cluster variables that arise.  Therefore
		\begin{displaymath}
			\cA(\x,B) = \CC\left[x_1,x_2,\frac{x_2+1}{x_1},\frac{x_1 + x_2 + 1}{x_1x_2},\frac{x_1+1}{x_2}\right] \subseteq \CC[x_1^{\pm1},x_2^{\pm1}].
		\end{displaymath}
    In fact, any four elements of $\{x_1,x_2,\frac{x_2+1}{x_1},\frac{x_1 + x_2 + 1}{x_1x_2},\frac{x_1+1}{x_2}\}$ are sufficient to generate $\cA(\x,B)$.  This observation readily generalizes to cluster algebras associated to any $2\times 2$ exchange matrix and leads ultimately to the following important result.
  \end{example}
	
	\begin{theorem}\label{th:Laurent phenomenon}
		For any initial seed $(\x,B)$ with $\x = (x_1,\ldots, x_n)$, the associated cluster algebra lies in the Laurent polynomial ring
		\begin{displaymath}
			\cA(\x,B) \subseteq \CC[x_1, x_1^{-1}, \ldots, x_n, x_n^{-1}].
		\end{displaymath}
		In particular, every cluster variable can be expressed as a Laurent polynomial in $x_1,\ldots, x_n$.
	\end{theorem}

  \begin{exercise}
    Compute all cluster variables and coefficient variables for cluster algebras associated to $B=\left[\begin{array}{cc} 0 & b\\ -c & 0\end{array}\right]$ with $b,c\in\ZZ_{>0}$ and $bc\le 3$.  Justify why attempting such a calculation is futile for $bc\ge4$.
  \end{exercise}

  \begin{exercise}
    Prove the Laurent phenomenon for rank 2 cluster algebras.  <Include guiding hints and make the question more explicit.>
  \end{exercise}
	
	\subsection{Increased generality}
	We generalize the previous definitions in two ways, first by allowing more general exchange matrices and then by allowing for certain coefficients in the exchange relations.
	
	\begin{definition}
		An $n \times n$ integer matrix $B$ is \emph{skew-symmetrizable} if there is a diagonal matrix $D$ with positive integer diagonal entries such that $DB$ is skew symmetric.
	\end{definition}
	
	As an example, a $2 \times 2$ matrix $B$ is skew-symmetrizable if and only if
	\begin{displaymath}
		B = \left[ \begin{array}{cc} 	0 & b \\	-c & 0 \\	\end{array}	\right]
		\textrm{ or } B = \left[ \begin{array}{cc} 	0 & -b \\	c & 0 \\	\end{array}	\right]
	\end{displaymath}
	for positive integers $b$ and $c$.  In either case, a possible symmetrizing matrix is
	\begin{displaymath}
		D = \left[ \begin{array}{cc} 	c & 0 \\	0 & b \\	\end{array}	\right]
	\end{displaymath}
	
	The second generalization is to allow so-called frozen variables that never mutate, but which play a part in the exchange relations for the cluster variables. An \emph{extended cluster}, by convention, is typically written $\x = (x_1,\ldots, x_n, x_{n+1},\ldots x_m)$ where $x_1,\ldots, x_n$ are the cluster variable and $x_{n+1},\ldots, x_m$ are the frozen variables.  An \emph{extended exchange matrix} is an $m \times n$ integer matrix $\tilde{B}$ with the property that its top $n \times n$ part is skew-symmetrizable.
	
	\begin{definition}
	Fix a seed $(\x, \tilde B)$ with $\x=(x_1,\ldots, x_m)$ an extended cluster and $\tilde{B}$ an extended exchange matrix.  For an integer $k = 1,2,\ldots, n$, the \emph{seed mutation} $\mu_k$ in direction $k$ produces a new seed $\mu_k(\x,\tilde{B}) = (\x',\tilde{B}')$ with $\x' = (x_1,\ldots, x_{k-1}, x_k', x_{k+1},\ldots, x_m)$.  The formulas for $x_k'$ and the entries $\tilde{b}'_{ij}$ of $B'$ are the same as in \eqref{eq:exchange relation} and \eqref{eq:matrix mutation}, where in the former the products now range from $1$ to $m$ instead of from $1$ to $n$.
	\end{definition}
	
	\begin{lemma}
	Lemma \ref{lem:mutate} holds in this generalized setting.  Moreover, the skew-symmetrizing matrix $D$ is unchanged by mutation.
	\end{lemma}
	
	Given a seed $(\x, \tilde{B})$ as above, the corresponding cluster algebra $\cA(\x, \tilde{B})$ is defined to be the subalgebra of $\CC(x_1,\ldots, x_m)$ generated by all cluster variables reachable from this seed together with the frozen variables (which appear in each seed).  Cluster algebras at this level of generality are refered to as cluster algebras of \emph{geometric type}.
	
	\begin{example}
		Let $n=3$ and $m=9$.  Denote the cluster 
		\begin{displaymath}
			\x = (\Delta_{13},\Delta_{14},\Delta_{15}, \Delta_{12},\Delta_{23},\Delta_{34},\Delta_{45},\Delta_{56},\Delta_{16})
		\end{displaymath}
		where the last $m-n=6$ variables are frozen.  Let
		\begin{displaymath}
			\tilde{B} = \left[\begin{array}{ccc}
			0 & 1 & 0 \\
			-1 & 0 & 1 \\
			0 & -1 & 0 \\
			1 & 0 & 0 \\
			-1 & 0 & 0 \\
			1 & -1 & 0 \\
			0 & 1 & -1 \\
			0 & 0 & 1 \\
			0 & 0 & -1 \\
			\end{array}
			\right].
		\end{displaymath}
		Let $\cA = \cA(\x,\tilde{B})$.  Interpret the initial variables as the indicated Plucker coordinates in $Gr_{2,6}$.
		
		\begin{subexercise} \ \newline
			\begin{itemize}
			\item The cluster variables of $\cA$ are precisely the $\Delta_{ij}$ with $1 \leq i < j \leq n$.
			\item The clusters of $\cA$ are precisely the $x_T$ (see Example \ref{example:Gr2n}) for $T$ a triangulation of a hexagon. 
			\end{itemize}
		\end{subexercise}
		For instance, the new cluster variable produced by applying $\mu_1$ to the initial seed is
		\begin{displaymath}
		\frac{\Delta_{12}\Delta_{34} + \Delta_{14}\Delta_{23}}{\Delta_{13}}
		\end{displaymath}
		which equals $\Delta_{24}$.
	\end{example}
	
	There is an alternate formulation of cluster algebras of geometric type which focuses on the roles the frozen variables $x_{n+1},\ldots, x_m$ play in the exchange relations, rather than on the variables themselves.  Let $(\x, \tilde{B})$ be an extended seed.  For $k=1,\ldots, n$ let 
	\begin{displaymath}
	y_k = \prod_{i=n+1}^m x_i^{b_{ik}}.
	\end{displaymath}
	Define an operation (called auxiliary addition) $\oplus$ on Laurent monomials by
	\begin{displaymath}
	\prod_{i=n+1}^m x_i^{e_i} \oplus \prod_{i=n+1}^m x_i^{f_i} = \prod_{i=n+1}^m x_i^{\min(e_i,f_i)}.
	\end{displaymath}
	Using this operation, we can extract positive and negative exponents as follows
	\begin{displaymath}
	\frac{y_k}{1 \oplus y_k} = \prod_{\substack{i=n+1,\ldots, m\\ b_{ik > 0}}} x_i^{b_{ik}}
	\quad \quad \quad \frac{1}{1 \oplus y_k} = \prod_{\substack{i=n+1,\ldots, m\\ b_{ik < 0}}} x_i^{-b_{ik}}.
	\end{displaymath}
	The change in going from the original exchange relation \eqref{eq:exchange relation} to the one in geometric type, then, can be summarized by saying that the two terms of the binomial are enriched with coefficients $y_k/(1 \oplus y_k)$ and $1/(1 \oplus y_k)$.  We now write
	\begin{displaymath} 
		x_k' = \frac{y_k\prod_{b_{ik}>0} x_i^{b_{ik}} + \prod_{b_{ik}<0} x_i^{-b_{ik}}}{(1 \oplus y_k)x_k}
	\end{displaymath}
	where the products are for $i$ from $1$ to $n$.
	
	\subsection{$Y$-patterns}
	There is an alternate version of seeds and mutations, closely related to the previous, which itself arises in many applications.  
	
	\begin{definition}
		A \emph{$Y$-seed} is a pair $(\y,B)$ consisting of an $n$-tuple $\y=(y_1,\ldots, y_n)$ of rational functions and an $n\times n$ skew symmetric matrix $B$.  For $k=1,\ldots, n$, the \emph{$Y$-seed mutation} $\mu_k$ is defined by
		\begin{displaymath}
			\mu_k((y_1,\ldots, y_n),B) = ((y_1',\ldots, y_n'),B')
		\end{displaymath}
		for $B'$ as defined in \eqref{eq:matrix mutation} and 
		\begin{equation}\label{eq:y mutation}
		% Warning: this seems to only agree with (2.3) in CA-IV in the skew-symmetric case
		y_j' = \begin{cases}
		y_k^{-1} & \text{if $j = k$;} \\
		y_jy_k^{|-b_{jk}|_+}(1+y_k)^{b_{jk}} & \text{if $j \neq k$.} \\
		\end{cases}
		\end{equation}
	\end{definition}
	
	In words, the $Y$-seed mutation $\mu_k$ has the following effects:
	\begin{enumerate}
	\item For each $j \neq k$, $y_j$ is multiplied by $(1+y_k)^{b_{jk}}$ if $b_{jk}>0$ and by $(1+y_k^{-1})^{b_{jk}}$ if $b_{jk}<0$,
	\item $y_k$ is inverted,
	\item $B$ is changed in the same way as for ordinary seed mutations.
	\end{enumerate}
	%The $Y$-dynamics also go by the name \emph{coefficient dynamics} because the tropicalisation of \cite{eq:y mutation} describes how the coefficients in a geometric type cluster algebra evolve.
	
	One connection between the two types of dynamics comes by way of a certain Laurent monomial change of variables.  Let $\tilde{B}$ be an $m \times n$, extended exchange matrix with $B$ its upper $n \times n$ part.  Given a seed $((x_1,\ldots, x_m),\tilde{B})$, define an associated $Y$-seed $((\hat{y}_1,\ldots, \hat{y}_n),B)$ by
	\begin{displaymath}
	\hat{y}_j = \prod_{i=1}^m x_i^{b_{ij}}
	\end{displaymath}
	
	\begin{proposition}
	Fix $k=1,\ldots, n$ and suppose $\mu_k(\x,\tilde{B}) = (\x',\tilde{B}')$.  Define $(y_1,\ldots, y_n)$ from $(\x,\tilde{B})$ and $(y_1',\ldots, y_n')$ from $(\x',\tilde{B}')$ as above.  Then 
	\begin{displaymath}
	\mu_k((\hat{y}_1,\ldots, \hat{y}_n), B) = ((\hat{y}_1',\ldots, \hat{y}_n'), B').
	\end{displaymath}
	\end{proposition}

%%%%%%%%%%%%%%%%%%%%%%%
\section{Basic Results}\label{sec:basic_theorems}
  As mentioned in the introduction cluster algebras have found applications in a surprising array of mathematical disciplines.  Much of their ubiquity comes from a number of remarkable theorems which we now explain.

  %Thus it is natural to seek a concrete understanding of these Laurent expansions of cluster variables.
  %\begin{theorem}
  %  Categorification?
  %\end{theorem}

  As we have seen in Examples~\ref{example:Gr2n} and~\ref{example:total positivity}, cluster algebras have roots in classical Lie theory.  A large number of objects in this realm are classified by Dynkin diagrams and cluster algebras are no exception.  Given an extended exchange matrix $\tilde B$ with principal square submatrix $B=(b_{ij})$, write $A:=A_{\tilde B}=(a_{ij})$ for the \emph{Cartan companion} of $\tilde B$ given by $a_{ij}=\begin{cases}2 & \text{if $i=j$;}\\ -|b_{ij}| & \text{otherwise.}\end{cases}$.  Note that the Cartan companion only depends on the principal part of $\tilde B$.
  \begin{theorem}
    A cluster algebra $\cA(\x,\tilde B)$ has only finitely many cluster variables if and only if there exists an extended exchange matrix mutation equivalent to $\tilde B$ whose Cartan companion is a finite-type Cartan matrix.
  \end{theorem}

  One motivation for the discovery of the cluster algebra formalism was the desire to find a combinatorial construction of dual canonical basis elements for (quantum) algebraic groups.  The dual canonical basis of a semisimple algebraic group induces bases on the coordinate rings of any of its subvarieties.  As we saw in Example~\ref{example:total positivity} there appears to be some kind of cluster algebra structure on the double Bruhat cells of a semisimple algebraic group, which one can hope will shed some light on the dual canonical basis.  In most cases the double Bruhat cell does not actually admit a cluster algebra structure but does admit a closely related structure.

  \begin{definition}
    Let $(\x,\tilde B)$ denote an extended seed.  Define the \emph{upper cluster algebra} $\cU(\x,\tilde B)$ as the intersection of all Laurent rings associated to extended seeds mutation equivalent to $(\x,\tilde B)$.
  \end{definition}

  \begin{remark}
    The Laurent phenomenon Theorem~\ref{th:Laurent phenomenon} guarantees the inclusion $\cA(\x,\tilde B)\subset\cU(\x,\tilde B)$, hence the name ``upper'' cluster algebra.
  \end{remark}

  \begin{theorem}
    The coordinate ring of any double Bruhat cell is an upper cluster algebra.
  \end{theorem}

  In some special cases, the cluster algebra coincides with its upper cluster algebra.
  \begin{definition}
    An exchange matrix $B=(b_{ij})$ is called \emph{acyclic} if there exists a permutation $\sigma$ so that $b_{\sigma_i\sigma_j}\le 0$ for $i<j$.
  \end{definition}
  This terminology can be easily understood in the case of a skew-symmetric exchange matrix, it is acyclic exactly when its associated quiver has no oriented cycles.

  \begin{theorem}
    Let $\cA(\x,\tilde B)$ be a cluster algebra where $\tilde B$ has full rank and acyclic principal part.  Then $\cA(\x,\tilde B)=\cU(\x,\tilde B)$.
  \end{theorem}

  Acyclicity also guarantees the existence of easily identifiable bases of a cluster algebra.  These are best understood by identifying the cluster algebra with another related algebra, its lower bound.
  \begin{definition}
    Let $(\x,\tilde B)$ be an extended seed and write $x'_k$ for the variable obtained by mutation in direction $k$.  The lower bound of $\cA(\x,\tilde B)$ at $(\x,\tilde B)$ is the subalgebra $\cL(\x,\tilde B)=\CC[x_1,x'_1,\ldots,x_n,x'_n,x_{n+1},\ldots,x_m]$.
  \end{definition}

  \begin{theorem}
    The cluster algebra $\cA(\x,\tilde B)$ coincides with its lower bound $\cL(\x,\tilde B)$ if and only if the principal part of $\tilde B$ is acyclic.
  \end{theorem}

  \begin{theorem}
    Positivity.
  \end{theorem}

  %The proof builds on a concrete combinatorial construction of cluster variables in rank 2.
  %\begin{theorem}
  %  Combinatorial constructions in rank 2.
  %\end{theorem}

%%%%%%%%%%%%%%%%%%%%%%%%%%%%%%%%%%%%%%%%%%%%%%%%%%%%%%%%
\section{Compatible Poisson Structures and Quantization}\label{sec:poisson_and_quantum}

  \begin{definition}
    A Poisson algebra $(A,\{\cdot,\cdot\})$ is an associative algebra $A$ equipped with an additional skew-symmetric binary operation (written as $\{x,y\}$ for $x,y\in A$) called the Poisson bracket for which the following holds:
    \begin{itemize}
      \item given any $x\in A$, the automorphism $\{x,\cdot\}:A\to A$, $y\mapsto\{x,y\}$ is a derivation with respect to both binary operations on $A$, i.e. for $x,y,z\in A$ we have
      \begin{equation}
        \{x,yz\}=\{x,y\}z+y\{x,z\}\quad\text{and}\quad\{x,\{y,z\}\}=\{\{x,y\},z\}+\{y,\{x,z\}\}.
      \end{equation}
    \end{itemize}
  \end{definition}

  \begin{example}
    A symplectic manifold $(M,\omega)$ is a smooth even-dimensional manifold $M^{2n}$ together with a non-degenerate 2-form $\omega\in H^2(M)$, i.e. $\omega^n\ne0$ is a volume form.  The symplectic structure $\omega$ provides a natural association of a vector field $\xi_f$ to a function $f\in\cO(M)$ via the contraction formula $\iota_{\xi_f}\omega=-df$.  The algebra of smooth functions on $M$ is then naturally a Poisson algebra via $\{f,g\}=\omega(\xi_f,\xi_g)$ for $f,g\in\cO(M)$.
  \end{example}
  More generally, any smooth manifold together with a Poisson structure on its algebra of regular functions is called a Poisson manifold.

  \begin{definition}
    Let $A$ be a Poisson algebra.  A finite collection $\bff\{f_i\}\subset A$ is called log-canonical if there exists a matrix $\Omega$ so that $\{f_i,f_j\}=\Omega_{ij}f_if_j$ for $f_i,f_j\in\bff$.
  \end{definition}
  This terminology is motivated by the property $\{\log f_i,\log f_j\}=\Omega_{ij}$ for $f_i,f_j\in\bff$.

  \begin{definition}
    Let $\cA$ be a cluster algebra.  A Poisson bracket $\{\cdot,\cdot\}$ on $\cA$ is compatible with the cluster algebra structure every cluster $\bx$ of $\cA$ is log-canonical with respect to $\{\cdot,\cdot\}$.
  \end{definition}

  \begin{lemma}
  \label{le:compatibility}
    Let $\cA(\bx,B)$ be a cluster algebra with compatible Poisson structure $\{\cdot,\cdot\}$.  Then writing $\Omega$ for the compatibility matrix, we have $B^T\Omega=[D\, \boldsymbol{0}]$ where $D$ is a diagonal matrix which skew-symmetrizes $B$.
  \end{lemma}
  \begin{exercise}
    Prove Lemma~\ref{le:compatibility}. 
  \end{exercise}

  \begin{theorem}
    A compatible Poisson structure exists if and only if the extended exchange matrix $\tilde B$ has full rank.  Moreover, the collection of all such Poisson structures is parametrized by an affine space of dimension $r(B)+{m\choose 2}$.
  \end{theorem}

  Given a cluster algebra $\cA$ of full rank, any choice of compatible Poisson structure gives rise to a quantization of $\cA$ as follows.

  \begin{definition}
    Quantum cluster algebra.
  \end{definition}

%%%%%%%%%%%%%%%%%%%%%%%%%%%%%%%%%%%%%%%%%%%%
\section{Applications to Integrable Systems}\label{sec:integrable_systems}
	\begin{definition}
		A \emph{quiver} is a directed graph without loops (arrows from a vertex to itself) and without oriented $2$-cyclets.
	\end{definition}
	
	A skew-symmetric $n\times n$ exchange matrix $B$ is often represented instead as a quiver $Q$ on vertex set $\{1,2,\ldots, n\}$.  More precisely, if $i,j$ are such that $b_{ij}>0$ then $Q$ has $b_{ij}$ arrows from $i$ to $j$ (and none from $j$ to $i$).  Matrix mutation on $B$ induces so-called quiver mutation on $Q$.
	
	\begin{definition}
		Given some $k=1,2,\ldots, n$, the \emph{quiver mutation} $\mu_k$ gives rise to $Q' = \mu_k(Q)$ by applying the following steps to $Q$:
		\begin{enumerate}
		\item for each pair of vertices $i,j$ and arrows $i\to k$ and $k \to j$, add an arrow $i \to j$,
		\item reverse all arrows $i \to k$ or $k \to i$,
		\item erase in turn pairs of arrows $i \to j$ and $j \to i$ to eliminate any $2$-cycles.
		\end{enumerate}
	\end{definition}
	
	Given a quiver $Q$ and a sequence $k_1,k_2,\ldots, k_l$ of its vertices (possibly with repeats), one can consider the dynamical system obtained by repeated applications of the composite mutation $\mu_{k_l} \circ \cdots \circ \mu_{k_2} \circ \mu_{k_1}$ starting from an initial seed $(\x, Q)$ or $Y$-seed $(\y,Q)$.  In the special case when the final quiver after one application equals the initial one, i.e.
	\begin{displaymath}
		\mu_{k_l} \circ \cdots \circ \mu_{k_2} \circ \mu_{k_1}(Q) = Q,
	\end{displaymath}
	the system amounts to iterating a fixed birational transformation.  It is natural to investigate these mappings for properties such as periodicity and integrability.

\subsection{Zamolodchikov periodicity}
	Fix positive integers $r$ and $s$ and consider the quiver $Q$ on vertex set $\{1,\ldots, r \} \times \{1,\ldots, s\}$ as illustrated in the case $r=4$, $s=3$ by
	
	\begin{pspicture}(5,4)
		\cnode(1,1){0.1}{v11}
		\cnode(2,1){0.1}{v21}
		\cnode(3,1){0.1}{v31}
		\cnode(4,1){0.1}{v41}
		\cnode(1,2){0.1}{v12}
		\cnode(2,2){0.1}{v22}
		\cnode(3,2){0.1}{v32}
		\cnode(4,2){0.1}{v42}
		\cnode(1,3){0.1}{v13}
		\cnode(2,3){0.1}{v23}
		\cnode(3,3){0.1}{v33}
		\cnode(4,3){0.1}{v43}
		\ncline{->}{v11}{v21}
		\ncline{->}{v31}{v21}
		\ncline{->}{v31}{v41}
		\ncline{<-}{v12}{v22}
		\ncline{<-}{v32}{v22}
		\ncline{<-}{v32}{v42}
		\ncline{->}{v13}{v23}
		\ncline{->}{v33}{v23}
		\ncline{->}{v33}{v43}
		\ncline{<-}{v11}{v12}
		\ncline{<-}{v13}{v12}
		\ncline{->}{v21}{v22}
		\ncline{->}{v23}{v22}
		\ncline{<-}{v31}{v32}
		\ncline{<-}{v33}{v32}
		\ncline{->}{v41}{v42}
		\ncline{->}{v43}{v42}
	\end{pspicture}
	
	Let $\mu_{\textrm{even}}$ be the compound mutation given by applying each $\mu_{i,j}$ with $i+j$ even in turn.  As no arrows connect these even vertices, it follows that these mutations commute so the order does not matter.  Define $\mu_{\textrm{odd}}$ similarly as the compound mutation at all odd vertices.
	
	\begin{exercise}
		Let $Y = (Y_{i,j})_{i=1,\ldots, r; j=1,\ldots, s}$.  Then $\mu_{\textrm{even}}(Y,Q) = (Y',-Q)$ where
		\begin{equation} \label{eq:YSystemCases}
			Y_{i,j}' = \begin{cases}
			Y_{i,j}^{-1} & \textrm{if $i+j$ even} \\
			Y_{i,j}\frac{\displaystyle\prod_{|i-i'|=1}(1+Y_{i',j})}{\displaystyle\prod_{|j-j'|=1}(1+Y_{i,j'}^{-1})} & \textrm{if $i+j$ odd} \\
			\end{cases}
		\end{equation}
		and $-Q$ denotes $Q$ with all its arrows reversed.
	\end{exercise}
	
	Now begin with the $Y$-seed $(Y_0,Q)$ and let 
	\begin{align*}
	(Y_1,-Q) &= \mu_{\textrm{even}}(Y_0,Q) \\ 
	(Y_2,Q) &= \mu_{\textrm{odd}}(Y_1,-Q) \\ 
	(Y_3,-Q) &= \mu_{\textrm{even}}(Y_2,Q) \\ 
	&\vdots
	\end{align*}
	where $Y_t = (Y_{ijt})_{i=1,\ldots, r; j=1,\ldots, s}$.  It follows from the above that
	\begin{displaymath}
		Y_{i,j,t+1} = \begin{cases}
		Y_{i,j,t}^{-1} & \textrm{if $i+j+t$ even} \\
		Y_{i,j,t}\frac{\displaystyle\prod_{|i-i'|=1}(1+Y_{i',j,t})}{\displaystyle\prod_{|j-j'|=1}(1+Y_{i,j',t}^{-1})} & \textrm{if $i+j+t$ odd} \\
		\end{cases}
	\end{displaymath}
	It is convenient to consider only the $Y_{i,j,t}$ with $i+j+t$ odd, which satisfy their own recurrence
	\begin{equation} \label{eq:YSystem}
		Y_{i,j,t-1}Y_{i,j,t+1} = \frac{\displaystyle\prod_{|j-j'|=1}(1+Y_{i,j',t}^{-1})}{\displaystyle\prod_{|i-i'|=1}(1+Y_{i',j,t})}
	\end{equation}
	for all $i+j+t$ even.  This recurrence is called the type $A_r \times A_s$ $Y$-system, a special case of a family of systems conjectured by Zamolodchikov to be periodic.  The proof, in this case, is due to A. Volkov \cite{V07}.
	
	\begin{theorem}
		The $Y$-system \eqref{eq:YSystem} on variables $Y_{i,j,t}$ with $i+j+t$ odd, $i=1,\ldots, r$, and $j=1,\ldots, s$ has period $2(r+s+2)$ in the $t$-direction, that is
		\begin{displaymath}
			Y_{i,j,t+2(r+s+2)} = Y_{i,j,t}.
		\end{displaymath}
	\end{theorem}
	
	Returning to the $Y$-pattern point of view, the initial seed $(Y_0,Q)$ consists of the $Y_{i,j,0}$ for $i+j$ odd and the $Y_{i,j,1}^{-1}$ for $i+j$ even.  The seed $(Y_2,Q) = \mu_{\textrm{odd}}(\mu_{\textrm{even}}(Y_0,Q))$ consists of the $Y_{i,j,2}$ for $i+j$ odd and the $Y_{i,j,3}^{-1}$ for $i+j$ even.  The periodicity theorem, then, asserts that the rational map $\mu_{\textrm{odd}} \circ \mu_{\textrm{even}}$ has order $r+s+2$.
	
	\subsection{The pentagram map} The pentagram map is a discrete dynamical system defined on the space of polygons in the projective plane.  Figure \ref{fig:pentagram} shows a polygon $A$ and the corresponding output $B=T(A)$, with $T$ being the notation for the pentagram map.  Each vertex of $B$ lies at the intersection of two consecutive ``shortest'' diagonals of $A$.
	
	\begin{figure} \label{fig:pentagram}
	\begin{pspicture}(5,5)
	\pspolygon[linewidth=2pt](1,2)(1,3)(2,4.5)(3,5)(4.5,4.5)(5,3.5)(4.5,2)(3.5,1)(2,1)
  \pspolygon[linestyle=dashed](1,2)(2,4.5)(4.5,4.5)(4.5,2)(2,1)(1,3)(3,5)(5,3.5)(3.5,1)
  \pspolygon[linewidth=2pt](1.22,2.55)(1.67,3.67)(2.5,4.5)(3.67,4.5)(4.5,3.87)(4.5,2.67)(3.97,1.79)(2.75,1.3)(1.62,1.75)
	\uput[ul](2,4.5){$A$}
	\uput[dr](1.67,3.67){$B=T(A)$}
	\end{pspicture}
	\caption{An application of the pentagram map}
	\end{figure}
	
	The pentagram map was introduced by R. Schwartz \cite{S92} and extensively studied by V. Ovsienko, Schwartz and S. Tabachnikov \cite{OST10, OST13}.  In particular, they demonstrated that the pentagram map is a complete integrable system.  Another take on integrability was provided by F. Soloviev \cite{S13}.  In the same span of time, M. Glick \cite{G11} described the pentagram map as mutations in a $Y$-pattern, building on work of Schwartz \cite{S08} who had found a lift to the octahedron recurrence.  Finally, M. Gekhtman, M. Shapiro, Tabachnikov and A. Vainshtein \cite{GSTV12} gave a uniform treatment of integrability and ``clusterness'' of the pentagram map, and generalizations thereof, in terms of weighted networks on tori.  A similar framework is provided by the cluster integrable systems of A. Goncharov and R. Kenyon \cite{GK13}, albeit without any explicit mention of the pentagram map.
	
	The connection between the pentagram map and $Y$-patterns comes by way of certain geometrically defined coordinates on the space of polygons.  The \emph{cross ratio} of $4$ real numbers $x_1,x_2,x_3,x_4$ is 
	\begin{displaymath}
		\chi(x_1,x_2,x_3,x_4) = \frac{(x_1-x_2)(x_3-x_4)}{(x_1-x_3)(x_2-x_4)}.
	\end{displaymath}
	The cross ratio is invariant under projective transformations, that is 
	\begin{displaymath}
		\chi(f(x_1),f(x_2),f(x_3),f(x_4)) = \chi(x_1,x_2,x_3,x_4)
	\end{displaymath}
	for any fractional linear map $f(x) = (ax+b)/(cx+d)$.  For this reason, there is a well-defined notion in the plane of the cross ratio of 4 collinear points or of 4 concurrent lines.
	
	Given an $n$-gon $A$, label its sides and vertices consecutively with the integers $\{1,2,\ldots, 2n\}$.  Such a labeling induces a canonical labeling on $T(A)$ as is illustrated in Figure \ref{fig:labeling}.  Note that the parities of the vertex labels and of the edge labels are interchanged by $T$.
	
	\begin{figure} 
%\psset{unit=1.5cm}
\begin{pspicture}(6,4)
	\rput(0,-0.5){
  % Original Pentagon
  \pspolygon[showpoints=true,linewidth=2pt](1,1)(4,1)(5,3)(3,4)(1,3)
	\uput[l](1,2){$1$}
  \uput[225](1,1){$2$}
	\uput[d](2.5,1){$3$}
  \uput[270](4,1){$4$}
	\uput[dr](4.5,2){$5$}
  \uput[0](5,3){$6$}
	\uput[ur](4,3.5){$7$}
  \uput[90](3,4){$8$}
	\uput[ul](2,3.5){$9$}
  \uput[l](1,3){$10$}
  % New Pentagon
  \pspolygon[linestyle=dashed](1,1)(5,3)(1,3)(4,1)(3,4)
  \pspolygon[showpoints=true,linewidth=2pt](1.92,2.38)(2.71,1.86)(3.57,2.28)(3.33,3)(2.33,3)
  \uput[l](1.92,2.38){$1$}
	\uput[dl](2.31,2.12){$2$}
	\uput[d](2.71,1.86){$3$}
	\uput[dr](3.14,2.07){$4$}
  \uput[330](3.57,2.28){$5$}
	\uput[20](3.45,2.64){$6$}
  \uput[ur](3.33,3){$7$}
	\uput[u](2.83,3){$8$}
  \uput[ul](2.33,3){$9$}
	\uput[160](2.12,2.69){$10$}
  
  }
\end{pspicture}
%\psset{unit=1cm}
\caption{Possible labelings of two polygons related by the pentagram map}
\label{fig:labeling}
\end{figure}

\begin{definition}
	The \emph{$y$-parameters} of a polygon $A$ are real numbers $y_1(A),\ldots, y_{2n}(A)$ defined by
	\begin{align*}
		y_i(A) = \begin{cases}
		-\left(\chi(\join{i}{(i-4)}, i-1, i+1, \join{i}{(i+4)})\right)^{-1} & \textrm{ if $i$ is a vertex of $A$} \\
		-\chi(\meet{i}{(i-4)}, i-1, i+1, \meet{i}{(i+4)}) & \textrm{ if $i$ is a side of $A$} \\
		\end{cases}
	\end{align*}
	This definition is illustrated in Figure \ref{fig:defy}.
\end{definition}

\begin{figure}
\psset{unit=.8cm}
\begin{pspicture}(12,6)
\rput(0,.5){
\psline(1,1)(3,1)(5,2)(5,3)(4,4)
\uput[d](1,1){$1$}
\uput[d](2,1){$2$}
\uput[d](3,1){$3$}
\uput[dr](4,1.5){$4$}
\uput[dr](5,2){$5$}
\uput[r](5,2.5){$6$}
\uput[r](5,3){$7$}
\uput[ur](4.5,3.5){$8$}
\uput[ur](4,4){$9$}
\psline[linewidth=2pt](1,1)(5,2)
\psline[linewidth=2pt](3,1)(5,2)
\psline[linewidth=2pt](5,3)(5,2)
\psline[linewidth=2pt](4,4)(5,2)
}
\rput(7,0){
\psline(1,1)(3,1)(5,2)(5,3)(4,4)(2,5)
\uput[ur](1,1){$1$}
\uput[ur](2,1){$2$}
\uput[u](3,1){$3$}
\uput[ul](4,1.5){$4$}
\uput[r](5,2){$5$}
\uput[r](5,2.5){$6$}
\uput[r](5,3){$7$}
\uput[dl](4.5,3.5){$8$}
\uput[dl](4,4){$9$}
\uput[d](3,4.5){$10$}
\uput[d](2,5){$11$}
\psline(3,1)(5,1)(5,3.5)(4,4)
\psdots[dotsize=2pt 3](5,1)(5,2)(5,3)(5,3.5)
}
\end{pspicture}
\psset{unit=1cm}
\caption{The cross ratios corresponding to the two types of $y$-parameters.  On the left, $-(y_5)^{-1}$ is the cross ratio of the lines $\join{5}{1}$, $4$, $6$, and $\join{5}{9}$.  On the right, $-y_6$ is the cross ratio of the points $\meet{6}{2}$, $5$, $7$, and $\meet{6}{10}$ .} 
\label{fig:defy}
\end{figure}
	
	\begin{proposition}
		Let $A$ be an $n$-gon with $y$-parameters $y_i = y_i(A)$.  Let $B= T(A)$ and denote its $y$-parameters $y_i' = y_i(B)$.  Then
		\begin{equation} \label{eq:T}
			y_i' = \begin{cases}
			y_i^{-1} & \textrm{if $i$ is a side of $B$} \\
			y_i\frac{(1+y_{i-1})(1+y_{i+1})}{(1+y_{i-3}^{-1})(1+y_{i+3}^{-1})} & \textrm{if $i$ is a vertex of $B$}
			\end{cases}
		\end{equation}
		In this equation, all indices are considered modulo $2n$.
	\end{proposition}
	
	The formula \eqref{eq:T} bares a distinct resemblance to \eqref{eq:YSystemCases}, so it is plausible that the former can also be described by $Y$-pattern mutations for an appropriate quiver.  The desired quiver $Q_n$ has vertex set $\{1,2,\ldots,2n\}$ and arrows $j \to (j\pm1)$ and $j \leftarrow (j \pm 3)$ for each odd $j$, with vertices considered modulo $2n$.  The quiver $Q_8$ is in Figure \ref{fig:GlickQuiver}.  Similar to before, define compound mutations
	\begin{align*}
		\mu_{\textrm{odd}} &= \mu_{2n-1} \circ \cdots \circ \mu_3 \circ \mu_1 \\
		\mu_{\textrm{even}} &= \mu_{2n} \circ \cdots \circ \mu_4 \circ \mu_2 
	\end{align*}
	
	\begin{figure}
\begin{pspicture}(-5,-5)(5,5)
\SpecialCoor
\cnode(2;22.5){.25}{v1}
\rput(v1){1}
\cnode(4;45){.25}{v2}
\rput(v2){2}
\cnode(2;67.5){.25}{v3}
\rput(v3){3}
\cnode(4;90){.25}{v4}
\rput(v4){4}
\cnode(2;112.5){.25}{v5}
\rput(v5){5}
\cnode(4;135){.25}{v6}
\rput(v6){6}
\cnode(2;157.5){.25}{v7}
\rput(v7){7}
\cnode(4;180){.25}{v8}
\rput(v8){8}
\cnode(2;202.5){.25}{v9}
\rput(v9){9}
\cnode(4;225){.25}{v10}
\rput(v10){10}
\cnode(2;247.5){.25}{v11}
\rput(v11){11}
\cnode(4;270){.25}{v12}
\rput(v12){12}
\cnode(2;292.5){.25}{v13}
\rput(v13){13}
\cnode(4;315){.25}{v14}
\rput(v14){14}
\cnode(2;337.5){.25}{v15}
\rput(v15){15}
\cnode(4;0){.25}{v16}
\rput(v16){16}

\psset{arrowsize=5pt}
\psset{arrowinset=0}
\ncline{->}{v1}{v16}   \ncline{->}{v1}{v2}
\ncline{->}{v3}{v2}    \ncline{->}{v3}{v4} 
\ncline{->}{v5}{v4}    \ncline{->}{v5}{v6}
\ncline{->}{v7}{v6}    \ncline{->}{v7}{v8}
\ncline{->}{v9}{v8}    \ncline{->}{v9}{v10}
\ncline{->}{v11}{v10}  \ncline{->}{v11}{v12}
\ncline{->}{v13}{v12}  \ncline{->}{v13}{v14}
\ncline{->}{v15}{v14}  \ncline{->}{v15}{v16}
\ncline{->}{v16}{v13} \ncline{->}{v16}{v3}
\ncline{->}{v2}{v15} \ncline{->}{v2}{v5}
\ncline{->}{v4}{v1} \ncline{->}{v4}{v7}
\ncline{->}{v6}{v3} \ncline{->}{v6}{v9}
\ncline{->}{v8}{v5} \ncline{->}{v8}{v11}
\ncline{->}{v10}{v7} \ncline{->}{v10}{v13}
\ncline{->}{v12}{v9} \ncline{->}{v12}{v15}
\ncline{->}{v14}{v11} \ncline{->}{v14}{v1}

\end{pspicture}
\caption{The quiver related to the action of the pentagram map on octagons} \label{fig:GlickQuiver}
\end{figure}

\begin{theorem} \label{thm:pentagram}
	Let $A$ be an $n$-gon with $y$-parameters $y_1,\ldots, y_{2n}$ and let $k$ be a positive integer.  Beginning from the $Y$-seed $((y_1,\ldots, y_{2n}), Q_n)$, apply $k$ compound mutations alternating between $\mu_{\textrm{even}}$ and $\mu_{\textrm{odd}}$.  The result will have the form $((y_1',\ldots, y_{2n}'), (-1)^kQ_n)$ where 
	\begin{displaymath}
	y_i' = y_i(T^k(A)). 
	\end{displaymath}
\end{theorem}

\begin{remark}
	It is natural to ask what values the $n$-tuple $(y_1(A),\ldots, y_{2n}(A))$ can take, and also, to what extend this data suffices to reconstruct $A$.  These questions are easier to answer once extending to a larger family of objects called twisted polygons.  In this setting, the $y_i$ satisfy a single relation $y_1\cdots y_{2n}=1$.  Moreover, the $y_i$ determine a twisted polygon uniquely up to projective equivalence and a one-parameter rescaling operation due to Schwartz \cite{S08}.  The $Y$-pattern dynamics described in Theorem \ref{thm:pentagram}, then, characterize the dynamics of the pentagram map on the space of twisted polygons modulo these equivalences.
\end{remark}
	
	We now give a somewhat informal presentation of integrability of the pentagram map from the cluster algebra perspective.  The first ingredient is a compatible Poisson structure.
	
	\begin{proposition}
		Define a Poisson bracket by $\{y_i,y_j\} = b_{ij}y_iy_j$ where $B$ is the exchange matrix associated to the pentagram quiver $Q_n$.  Then this bracket is preserved by the pentagram map, i.e.
		\begin{displaymath}
			\{f,g\} = \{f\circ T, g \circ T\}
		\end{displaymath}
		for all functions $f$ and $g$.
	\end{proposition}
	
	Some more work is needed to get at the conserved quantities.  The quiver $Q_n$ can be embedded on a torus in such a way that, when circling any vertex, the arrows alternate between incoming and outgoing.  The faces are all quadrilaterals and come in two types, $(i-2)\to (i-1) \to (i+2) \to (i+1) \to (i-2)$ for $i$ odd and $(i-2) \to (i+1) \to (i+2) \to (i-1) \to (i-2)$ for $i$ even.  Let $G = (V,E)$ be the dual graph with $V = \{1,2,\ldots, 2n\}$ where vertex $i$ corresponds to the face described in the previous sentence.  Coincidentally, $G$ is identical to $Q_n$ except without orientations, so $\overline{ij} \in E$ if and only if $i-j \equiv \pm 1, \pm 3 \pmod{2n}$.  Figure \ref{fig:torus} shows the graph $G$ on the torus in the case $n=4$, with the vertices $1,2,\ldots, 8$ appearing in order from left to right.
	
	\begin{figure}
	\begin{pspicture}(-1,0)(10,2)
\cnode(1,1){.08}{v1}
\cnode*(2,1){.08}{v2}
\cnode(3,1){.08}{v3}
\cnode*(4,1){.08}{v4}
\cnode(5,1){.08}{v5}
\cnode*(6,1){.08}{v6}
\cnode(7,1){.08}{v7}
\cnode*(8,1){.08}{v8}
\pnode(.5,1){w1}
\pnode(8.5,1){e8}
\pnode(0,1.5){n0}
\pnode(1,1.5){n1}
\pnode(2,1.5){n2}
\pnode(3,1.5){n3}
\pnode(4,1.5){n4}
\pnode(5,1.5){n5}
\pnode(6,1.5){n6}
\pnode(7,1.5){n7}
\pnode(8,1.16){n8}
\pnode(9,.83){n9}
\pnode(0,1.16){s0}
\pnode(1,.83){s1}
\pnode(2,.5){s2}
\pnode(3,.5){s3}
\pnode(4,.5){s4}
\pnode(5,.5){s5}
\pnode(6,.5){s6}
\pnode(7,.5){s7}
\pnode(8,.5){s8}
\pnode(9,.5){s9}
\pspolygon[linestyle=dashed](-1,1.5)(7,1.5)(10,.5)(2,.5)
\ncline{w1}{v1}
\ncline{v1}{v2}
\ncline{v2}{v3}
\ncline{v3}{v4}
\ncline{v4}{v5}
\ncline{v5}{v6}
\ncline{v6}{v7}
\ncline{v7}{v8}
\ncline{v8}{e8}
\ncline{v1}{n1}
\ncline{v2}{n2}
\ncline{v3}{n3}
\ncline{v4}{n4}
\ncline{v5}{n5}
\ncline{v6}{n6}
\ncline{v7}{n7}
\ncline{v8}{n8}
\ncline{v1}{s1}
\ncline{v2}{s2}
\ncline{v3}{s3}
\ncline{v4}{s4}
\ncline{v5}{s5}
\ncline{v6}{s6}
\ncline{v7}{s7}
\ncline{v8}{s8}
\ncline{s0}{n0}
\ncline{s9}{n9}
\end{pspicture}
	\caption{A bipartite graph on the torus dual to the quiver $Q_4$.}
	\label{fig:torus}
	\end{figure}
	
	The \emph{corner invariants} $x_1,\ldots, x_{2n}$ of an $n$-gon $A$ are certain geometrically defined quantities similar in spirit to the $y$-parameters.  In fact there is a simple relation
	\begin{displaymath}
		y_i = \begin{cases}
		-(x_ix_{i+1})^{-1} & \textrm{if $i$ is a vertex of $A$} \\
		-x_ix_{i+1} & \textrm{if $i$ is a side of $A$} \\
		\end{cases}
	\end{displaymath}
	which explains the relation $y_1y_2\cdots y_{2n}=1$.  Assign edge weights to $G$ as follows: each ``horizontal'' edge $\overline{i(i+1)}$ gets weight $1$ and each ``vertical edge'' $\overline{i(i+3)}$ gets weight $(-1)^ix_{i+2}$.  
	
	A \emph{perfect matching} of $G$ is a collection $M$ of its edges such that each $i \in V$ is an endpoint of exactly one $e \in M$.  The \emph{weight} of $M$, denoted $wt(M)$, is the product of the weights of its edges.  For instance, 
	\begin{displaymath}
		M = \{\overline{12}, \overline{36}, \overline{47}, \overline{58}\}
	\end{displaymath}
	is a perfect matching of $G_4$ and its weight is $x_5x_6x_7$.
	
	\begin{proposition}
		The sum
		\begin{displaymath}
			\sum_M wt(M)
		\end{displaymath}
		over perfect matchings $M$ of $G_n$ is a conserved quantity of the pentagram map acting on $n$-gons.  Moreover, there is a refinement of the set of perfect matchings breaking this sum into pieces, each of which is itself conserved, and together providing a complete family of integrals for the pentagram map.
	\end{proposition}
	
%%%%%%%%%%%%%%%%%%%%%%%%%%%
\begin{thebibliography}{99}
	\bibitem{dFK10} P. DiFrancesco and R. Kedem, $Q$-systems, heaps, paths and cluster positivity, \textsl{Comm. Math. Phys.} \textbf{293} (2010), 727--802.
	\bibitem{FG06} V. Fock and A. Goncharov, Moduli spaces of local systems and higher Teichmüller theory, \textsl{Publ. Math. Inst. Hautes Études Sci.} \textbf{103} (2006), 1--211.
	\bibitem{FST08} S. Fomin, M. Shapiro, and D. Thurston, Cluster algebras and triangulated surfaces. I. Cluster complexes, \textsl{Acta Math.} \textbf{201} (2008), 83--146.
	\bibitem{FZ02} S. Fomin and A. Zelevinsky, Cluster algebras I: Foundations, \textsl{J. Amer. Math. Soc.} \textbf{15} (2002), 497--529.
	\bibitem{FZ00} S. Fomin and A. Zelevinsky, Total positivity: tests and parametrizations, \textsl{Math. Intelligencer} \textbf{22} (2000), 23--33.
	\bibitem{GLS06} C. Geiss, B Leclerc, and J. Schroer, Rigid modules over preprojective algebras, \textsl{Invent. Math} \textbf{165} (2006), 589--632.
	\bibitem{GSV03} M. Gekhtman, M. Shapiro, and A. Vainshtein,  Cluster algebras and Poisson geometry, \textsl{Mosc. Math. J.} \textbf{3} (2003), 899--934.
	\bibitem{GSTV12} M. Gekhtman, M. Shapiro, S. Tabachnikov, and A. Vainshtein, Higher pentagram maps, weighted directed
 networks, and cluster dynamics, \textsl{Electron. Res. Announc. Math. Sci.} \textbf{19} (2012), 1--17.
	\bibitem{G11} M. Glick, The pentagram map and Y-patterns, \textsl{Adv. Math.} \textbf{227} (2011), 1019--1045.
	\bibitem{GK13} A. Goncharov and R. Kenyon, Dimers and cluster integrable systems, \textsl{Ann. Sci. \'Ec. Norm. Sup\'er.} \textbf{46} (2013), 747--813.
	\bibitem{MP07} G. Musiker and J. Propp, Combinatorial interpretations for rank-two cluster algebras of affine type, \textsl{ Electron. J. Combin.} \textbf{14} (2007), 23 pp.
	\bibitem{MSW13} G. Muisker, R. Schiffler, and L. Williams, Bases for cluster algebras from surfaces, \textsl{Compos. Math} \textbf{149} (2013), 2891--2944.
	\bibitem{OST10} V. Ovsienko, R. Schwartz, and S. Tabachnikov, The pentagram map: a discrete integrable system, \textsl{Comm. Math. Phys.} \textbf{299} (2010), 409-446.
	\bibitem{OST13} V. Ovsienko, R. Schwartz, and S. Tabachnikov, Liouville-Arnold integrability of the pentagram map on closed polygons, \textsl{Duke Math. J.} \textbf{162} (2013), 2149--2196.
	\bibitem{S92} R. Schwartz, The pentagram map, \textsl{Experiment. Math.} \textbf{1} (1992), 71--81.
	\bibitem{S08} R. Schwartz, Discrete monodromy, pentagrams, and the method of condensation, \textsl{J. Fixed Point Theory Appl
.} \textbf{3} (2008), 379--409.
	\bibitem{S13} F. Soloviev, Integrability of the Pentagram Map, \textsl{Duke Math. J.} \textbf{162} (2013), 2815--2853.
	\bibitem{V07} A. Volkov, On the periodicity conjecture for $Y$-systems, \textsl{Comm. Math. Phys.} \textbf{276} (2007), 509--517.
\end{thebibliography}

\end{document}